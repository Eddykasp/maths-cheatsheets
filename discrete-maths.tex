% !TeX spellcheck = en_GB
\documentclass[a4paper, 10pt]{article}

\usepackage{fontspec}
\usepackage{amsmath, amssymb}
\usepackage{multicol}
\usepackage{color}
\usepackage{colortbl}
\usepackage{logicproof}
\defaultfontfeatures{Mapping=tex-text,Scale=MatchLowercase}
\setmainfont{Ubuntu Light}

\title{Discrete Mathematics}
\author{Max Kasperowski}

\begin{document}
\maketitle
\tableofcontents

\newpage
\section{Logic}

\subsection[Propositional Logic]{Propositional Logic{\large ---Zeroth Order Logic}}
In propositional logic, propositions are denoted by letters (\(p, q\)) and are formed by connecting other propositions using logical connectives. Propositions can either be true (\(T\)) or false (\(F\)).
\paragraph{Logical Connectives}
The logical connectives listed below are the basic connectives available in propositional logic in order of their precedence. Below are the truthtables corresponding to each of the connectives.
\begin{enumerate}
    \item \( \neg \), not
    \item \( \land \), and, \( \bigwedge\limits_{i=1}^n p_i \)
    \item \( \lor \), or, \( \bigvee\limits_{i=1}^n p_i \)
    \item \( \rightarrow ,\Rightarrow \), implies (only if) defined as: \( p\rightarrow q \equiv \neg p\lor q \)
    \item \( \leftrightarrow ,\Leftrightarrow \), is equivalent to (if and only if, iff) defined as: \(p\leftrightarrow q \equiv (p\rightarrow q)\land (q\rightarrow p) \)
\end{enumerate}

\begin{multicols}{3}
\paragraph{Negation}\mbox{}\\
\begin{tabular}{@{ }c | c@{ }@{ }c}
p & \( \neg \) & p\\
\hline
T & \textcolor{red}{F} & T\\
F & \textcolor{red}{T} & F\\
\end{tabular}
\paragraph{Logical And}\mbox{}\\
\begin{tabular}{@{ }c@{ }@{ }c | c@{ }@{ }c@{ }@{ }c@{ }@{ }c@{ }@{ }c}
p & q &  & p & \( \land\) & q & \\
\hline
T & T &  & T & \textcolor{red}{T} & T & \\
T & F &  & T & \textcolor{red}{F} & F & \\
F & T &  & F & \textcolor{red}{F} & T & \\
F & F &  & F & \textcolor{red}{F} & F & \\
\end{tabular}
\paragraph{Logical Or}\mbox{}\\
\begin{tabular}{@{ }c@{ }@{ }c | c@{ }@{ }c@{ }@{ }c@{ }@{ }c@{ }@{ }c}
p & q &  & p & \( \lor \) & q & \\
\hline
T & T &  & T & \textcolor{red}{T} & T & \\
T & F &  & T & \textcolor{red}{T} & F & \\
F & T &  & F & \textcolor{red}{T} & T & \\
F & F &  & F & \textcolor{red}{F} & F & \\
\end{tabular}
\paragraph{Implication}\mbox{}\\
\begin{tabular}{@{ }c@{ }@{ }c | c@{ }@{ }c@{ }@{ }c@{ }@{ }c@{ }@{ }c}
p & q &  & p & \(\rightarrow\) & q & \\
\hline
T & T &  & T & \textcolor{red}{T} & T & \\
T & F &  & T & \textcolor{red}{F} & F & \\
F & T &  & F & \textcolor{red}{T} & T & \\
F & F &  & F & \textcolor{red}{T} & F & \\
\end{tabular}
\paragraph{Equivalence}\mbox{}\\
\begin{tabular}{@{ }c@{ }@{ }c | c@{ }@{ }c@{ }@{ }c@{ }@{ }c@{ }@{ }c}
p & q &  & p & \(\leftrightarrow\) & q & \\
\hline
T & T &  & T & \textcolor{red}{T} & T & \\
T & F &  & T & \textcolor{red}{F} & F & \\
F & T &  & F & \textcolor{red}{F} & T & \\
F & F &  & F & \textcolor{red}{T} & F & \\
\end{tabular}
\end{multicols}
\subsubsection{Definitions}
\paragraph{Converse, Contrapositive, Inverse}
When given the proposition \( p\rightarrow q \), \( q\rightarrow p \) is its converse, \( \neg q\rightarrow \neg p \) is its contrapositive and \( \neg p\rightarrow \neg q \) is its inverse. The contrapositive is equivalent to the original proposition and the converse and inverse are also equivalent.

\paragraph{Tautology}
A proposition that is always true (\(p\lor\neg p\)).

\paragraph{Contradiction}
A proposition that is always false (\(p\land\neg p\)).

\paragraph{Contingency}
A proposition that is neither a tautology nor a contradiction.

\paragraph{Logical Equivalence}
\(p\) and \(q\) are logically equivalent if \(p\leftrightarrow q\) is a tautology. The notation for equivalence is typically \(\equiv\).
\newpage
\subsubsection{Properties}
\begin{multicols}{2}
    \subparagraph{De Morgan's Laws}
    \[ \neg(p\land q) \equiv \neg p \lor \neg q \]
    \[ \neg(p\lor q) \equiv \neg p \land \neg q \]

    \subparagraph{Identity Laws}
    \[ p \land T \equiv p \]
    \[ p \lor F \equiv p \]

    \subparagraph{Domination Laws}
    \[ p \land F \equiv F \]
    \[ p \lor T \equiv T \]

    \subparagraph{Idempotent Laws}
    \[p \land p \equiv p\]
    \[ p \lor p \equiv p\]

    \subparagraph{Negation Laws}
    \[ p \land \neg p \equiv F \]
    \[ p \lor \neg p \equiv T \]

    \subparagraph{Commutative Laws}
    \[ p \land q \equiv q \land p\]
    \[ p \lor q \equiv q \lor p \]

    \subparagraph{Associative Laws}
    \[ (p \land q)\land r \equiv p\land(q\land r) \]
    \[ (p \lor q)\lor r \equiv p\lor(q\lor r) \]

    \subparagraph{Distributive Laws}
    \[ p\lor (q \land r) \equiv (p\lor q) \land (p\lor r) \]
    \[ p\land (q\lor r) \equiv (p\land q)\lor(p\land r) \]

    \subparagraph{Absorption Laws}
    \[ p \lor (p\land q) \equiv p \]
    \[ p \land (p\lor q) \equiv p \]
\end{multicols}

\subsubsection{Equivalence Proof}
This is an example of how to perform an equivalence proof. The aim is to show that \( \neg(p\lor (\neg p \land q)) \) is logically equivalent to \( \neg p\land \neg q \). We can prove this by forming a series of logical equivalences.
\begin{align*}
    \textcolor{red}{\neg (p\lor(\neg p\land q))} &\equiv \neg p \land \neg(\neg p\land q) & \text{2\textsuperscript{nd} De Morgan's law}\\
    \neg p \land \neg(\neg p\land q) &\equiv \neg p \land (p\lor\neg q) & \text{1\textsuperscript{st} De Morgan's law} \\
    \neg p \land (p\lor\neg q) &\equiv (\neg p \land p)\lor(\neg p\land \neg q) & \text{Associative law} \\
    (\neg p \land p)\lor(\neg p\land \neg q) &\equiv F \lor (\neg p \land \neg q) & \text{Negation law} \\
    F \lor (\neg p \land \neg q) &\equiv \textcolor{red}{\neg p \land \neg q} & \text{Identity law}
\end{align*}

\newpage
\subsection[Predicate Logic]{Predicate Logic {\large ---First Order Logic}}
Predicate logic uses quantified variables over non-logical objects and allows the use of sentences that contain variables. This allows a generalisation of propositions for a set of variables from a domain.

\subsubsection{Definitions}
\paragraph{Predicates}
A predicate is a generalisation of propositions when the variable \(x\) is replaced by a specific element from its domain. \(P(x)\) becomes a proposition. When no other domain is specified the domain is \(U\).

\paragraph{Quantifiers}
Quatifiers are used to express that a proposition is true for all elements of the domain and that there exists some element in the domain for which it is true. They also have the highest precedence among the logical operators.
\begin{align*}
    \text{Universal quantifier\quad\quad\quad\quad} & \textcolor{red}{\forall} xP(x) & P(x)\text{ is true for every \(x\) in \(U\)} \\
    \text{Existential quantifier\quad\quad\quad\quad} & \textcolor{red}{\exists} xP(x) & P(x)\text{ is true for some \(x\) in \(U\)}
\end{align*}

\subsubsection{Properties}
\paragraph{Uniqueness Quantifier}
The uniqueness quantifier is a commonly used quantifier to express that there is only one \(x\) for which \(P(x)\) is true. It is usually written as \(\exists !\) or \(\exists_1\).
\[\textcolor{red}{\exists_1}xP(x)\equiv\exists x(P(x)\land\forall y(P(y)\rightarrow y=x))\]
\paragraph{De Morgan's Laws}
De Morgan's laws for quantifiers state that \(P(x)\) is not true for all \(x\) if and only if there exists an \(x\) for which \(P(x)\) is false and furthermore that if \(P(x)\) is false for all \(x\) if and only if there does not exist an \(x\) for which \(P(x)\) is true.
\begin{align*}
    \neg\forall x P(x) &\equiv \exists x\neg P(x)\\
    \forall x\neg P(x) &\equiv \neg\exists x P(x)
\end{align*}

\subsection{Logical Proofs}
Using logical inference it is possible to build an argument given a set of premises to reach a logical conclusion. An argument is valid if truth of all premises (\(p_i)\)) implies that the conclusion \(q\) is also true.
\[\left(\bigwedge_{i=1}^n p_i\right)\rightarrow q \equiv T\]

\subsubsection{Rules of Inference}
\begin{multicols}{3}
\paragraph{Modus Ponens}\mbox{}\\
\begin{tabular}{c@{\,}l@{}}
                & \(p\rightarrow q\) \\
                & \(p\) \\\cline{2-2}
\(\therefore\)  & \(q\)
\end{tabular}

\paragraph{Modus Tollens}\mbox{}\\
\begin{tabular}{c@{\,}l@{}}
                & \(p\rightarrow q\) \\
                & \(\neg q\) \\\cline{2-2}
\(\therefore\)  & \(\neg p\)
\end{tabular}

\paragraph{Hypothetical Syllogism}\mbox{}\\
\begin{tabular}{c@{\,}l@{}}
                & \(p\rightarrow q\) \\
                & \(q\rightarrow r\) \\\cline{2-2}
\(\therefore\)  & \(p\rightarrow r\)
\end{tabular}

\paragraph{Disjunctive Syllogism}\mbox{}\\
\begin{tabular}{c@{\,}l@{}}
                & \(p\lor q\) \\
                & \(\neg p\) \\\cline{2-2}
\(\therefore\)  & \(q\)
\end{tabular}

\paragraph{Addition}\mbox{}\\
\begin{tabular}{c@{\,}l@{}}
                & \(p\) \\\cline{2-2}
\(\therefore\)  & \(p\lor q\)
\end{tabular}

\paragraph{Simplification}\mbox{}\\
\begin{tabular}{c@{\,}l@{}}
                & \(p\land q\) \\\cline{2-2}
\(\therefore\)  & \(p\)
\end{tabular}
\begin{tabular}{c@{\,}l@{}}
                & \(p\land q\) \\\cline{2-2}
\(\therefore\)  & \(q\)
\end{tabular}

\paragraph{Conjunction}\mbox{}\\
\begin{tabular}{c@{\,}l@{}}
                & \(p\) \\
                & \(q\) \\\cline{2-2}
\(\therefore\)  & \(p\land q\)
\end{tabular}

\paragraph{Resolution}\mbox{}\\
\begin{tabular}{c@{\,}l@{}}
                & \(\neg p\lor r\) \\
                & \(p\lor q\) \\\cline{2-2}
\(\therefore\)  & \(q\lor r\)
\end{tabular}


\end{multicols}

\end{document}
