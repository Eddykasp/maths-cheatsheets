% !TeX spellcheck = en_GB
\documentclass[a4paper, 10pt]{article}

\usepackage{fontspec}
\usepackage{amsmath}
\usepackage{multicol}
\usepackage{color}
\defaultfontfeatures{Mapping=tex-text,Scale=MatchLowercase}
\setmainfont{Ubuntu Light}

\title{Discrete Mathematics}
\author{Max Kasperowski}

\begin{document}
\maketitle
\tableofcontents

\newpage
\section[Propositional Logic]{Propositional Logic{\large ---Zeroth Order Logic}}
In propositional logic, propositions are denoted by letters (\(p, q\)) and are formed by connecting other propositions using logical connectives. Propositions can either be true (\(T\)) or false (\(F\)).
\subsection{Logical Connectives}
The logical connectives listed below are the basic connectives available in propositional logic in order of their precedence. Below are the truthtables corresponding to each of the connectives.
\begin{enumerate}
    \item \( \neg \), not
    \item \( \land \), and, \( \bigwedge\limits_{i=1}^n p_i \)
    \item \( \lor \), or, \( \bigvee\limits_{i=1}^n p_i \)
    \item \( \rightarrow ,\Rightarrow \), implies (only if) defined as: \( p\rightarrow q \equiv \neg p\lor q \)
    \item \( \leftrightarrow ,\Leftrightarrow \), is equivalent to (if and only if, iff) defined as: \(p\leftrightarrow q \equiv (p\rightarrow q)\land (q\rightarrow p) \)
\end{enumerate}

\begin{multicols}{2}
\paragraph{Negation}
\begin{tabular}{@{ }c | c@{ }@{ }c}
p & \( \neg \) & p\\
\hline
T & \textcolor{red}{F} & T\\
F & \textcolor{red}{T} & F\\
\end{tabular}
\paragraph{Logical And}
\begin{tabular}{@{ }c@{ }@{ }c | c@{ }@{ }c@{ }@{ }c@{ }@{ }c@{ }@{ }c}
p & q &  & p & \( \land\) & q & \\
\hline
T & T &  & T & \textcolor{red}{T} & T & \\
T & F &  & T & \textcolor{red}{F} & F & \\
F & T &  & F & \textcolor{red}{F} & T & \\
F & F &  & F & \textcolor{red}{F} & F & \\
\end{tabular}
\paragraph{Logical Or}
\begin{tabular}{@{ }c@{ }@{ }c | c@{ }@{ }c@{ }@{ }c@{ }@{ }c@{ }@{ }c}
p & q &  & p & \( \lor \) & q & \\
\hline
T & T &  & T & \textcolor{red}{T} & T & \\
T & F &  & T & \textcolor{red}{T} & F & \\
F & T &  & F & \textcolor{red}{T} & T & \\
F & F &  & F & \textcolor{red}{F} & F & \\
\end{tabular}
\paragraph{Implication}
\begin{tabular}{@{ }c@{ }@{ }c | c@{ }@{ }c@{ }@{ }c@{ }@{ }c@{ }@{ }c}
p & q &  & p & \(\rightarrow\) & q & \\
\hline
T & T &  & T & \textcolor{red}{T} & T & \\
T & F &  & T & \textcolor{red}{F} & F & \\
F & T &  & F & \textcolor{red}{T} & T & \\
F & F &  & F & \textcolor{red}{T} & F & \\
\end{tabular}
\paragraph{Equivalence}
\begin{tabular}{@{ }c@{ }@{ }c | c@{ }@{ }c@{ }@{ }c@{ }@{ }c@{ }@{ }c}
p & q &  & p & \(\leftrightarrow\) & q & \\
\hline
T & T &  & T & \textcolor{red}{T} & T & \\
T & F &  & T & \textcolor{red}{F} & F & \\
F & T &  & F & \textcolor{red}{F} & T & \\
F & F &  & F & \textcolor{red}{T} & F & \\
\end{tabular}
\end{multicols}
\subsection{Definitions}
When given the proposition \( p\rightarrow q \), \( q\rightarrow p \) is its converse, \( \neg q\rightarrow \neg p \) is its contrapositive and \( \neg p\rightarrow \neg q \) is its inverse. The contrapositive is equivalent to the original proposition and the converse and inverse are also equivalent.

\paragraph{Tautology}
A proposition that is always true (\(p\lor\neg p\)).

\paragraph{Contradiction}
A proposition that is always false (\(p\land\neg p\)).

\paragraph{Contingency}
A proposition that is neither a tautology nor a contradiction.

\paragraph{Logical Equivalence}
\(p\) and \(q\) are logically equivalent if \(p\leftrightarrow q\) is a tautology. The notation for equivalence is typically \(\equiv\).

\newpage
\subsection{Properties}
\begin{multicols}{2}
    \paragraph{De Morgan's Laws}
    \[ \neg(p\land q) \equiv \neg p \lor \neg q \]
    \[ \neg(p\lor q) \equiv \neg p \land \neg q \]

    \paragraph{Identity Laws}
    \[ p \land T \equiv p \]
    \[ p \lor F \equiv p \]

    \paragraph{Domination Laws}
    \[ p \land F \equiv F \]
    \[ p \lor T \equiv T \]

    \paragraph{Idempotent Laws}
    \[p \land p \equiv p\]
    \[ p \lor p \equiv p\]

    \paragraph{Negation Laws}
    \[ p \land \neg p \equiv F \]
    \[ p \lor \neg p \equiv T \]

    \paragraph{Commutative Laws}
    \[ p \land q \equiv q \land p\]
    \[ p \lor q \equiv q \lor p \]

    \paragraph{Associative Laws}
    \[ (p \land q)\land r \equiv p\land(q\land r) \]
    \[ (p \lor q)\lor r \equiv p\lor(q\lor r) \]

    \paragraph{Distributive Laws}
    \[ p\lor (q \land r) \equiv (p\lor q) \land (p\lor r) \]
    \[ p\land (q\lor r) \equiv (p\land q)\lor(p\land r) \]

    \paragraph{Absorption Laws}
    \[ p \lor (p\land q) \equiv p \]
    \[ p \land (p\lor q) \equiv p \]
\end{multicols}

\subsection{Equivalence Proof}
This is an example of how to perform an equivalence proof. The aim is to show that \( \neg(p\lor (\neg p \land q)) \) is logically equivalent to \( \neg p\land \neg q \). We can prove this by forming a series of logical equivalences.
\begin{align}
    \textcolor{red}{\neg (p\lor(\neg p\land q))} &\equiv \neg p \land \neg(\neg p\land q) & \text{2\textsuperscript{nd} De Morgan's law}\\
    \neg p \land \neg(\neg p\land q) &\equiv \neg p \land (p\lor\neg q) & \text{1\textsuperscript{st} De Morgan's law} \\
    \neg p \land (p\lor\neg q) &\equiv (\neg p \land p)\lor(\neg p\land \neg q) & \text{Associative law} \\
    (\neg p \land p)\lor(\neg p\land \neg q) &\equiv F \lor (\neg p \land \neg q) & \text{Negation law} \\
    F \lor (\neg p \land \neg q) &\equiv \textcolor{red}{\neg p \land \neg q} & \text{Identity law}
\end{align}

\section[Predicate Logic]{Predicate Logic {\large ---First Order Logic}}

\end{document}
