% !TeX spellcheck = en_GB
\documentclass[a4paper, 10pt]{article}

\usepackage{fontspec}
\usepackage{amsmath}
\usepackage{multicol}
\defaultfontfeatures{Mapping=tex-text,Scale=MatchLowercase}
\setmainfont{Ubuntu Light}

\title{Discrete Mathematics}
\author{Max Kasperowski}

\begin{document}
\maketitle
\tableofcontents

\newpage
\section{Propositional Logic}
In propositional logic, propositions are denoted by letters (\(p, q\)) and are formed by connecting other propositions using logical connectives. Propositions can either be true (\(T\)) or false (\(F\)).
\subsection{Logical Connectives}
The logical connectives listed below are the basic connectives available in propositional logic.
\begin{enumerate}
    \item \( \neg \), not
    \item \( \land \), and
    \item \( \lor \), or
    \item \( \Rightarrow \), implies (only if)
    \item \( \Leftrightarrow \), is equivalent to (if and only if, iff)
\end{enumerate}

\end{document}
