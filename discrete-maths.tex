% !TeX spellcheck = en_GB
\documentclass[a4paper, 10pt]{article}

\usepackage{fontspec}
\usepackage{amsmath}
\usepackage{multicol}
\usepackage{color}
\defaultfontfeatures{Mapping=tex-text,Scale=MatchLowercase}
\setmainfont{Ubuntu Light}

\title{Discrete Mathematics}
\author{Max Kasperowski}

\begin{document}
\maketitle
\tableofcontents

\newpage
\section{Propositional Logic}
In propositional logic, propositions are denoted by letters (\(p, q\)) and are formed by connecting other propositions using logical connectives. Propositions can either be true (\(T\)) or false (\(F\)).
\subsection{Logical Connectives}
The logical connectives listed below are the basic connectives available in propositional logic in order of their precedence. Below are the truthtables corresponding to each of the connectives.
\begin{enumerate}
    \item \( \neg \), not
    \item \( \land \), and
    \item \( \lor \), or
    \item \( \rightarrow ,\Rightarrow \), implies (only if)
    \item \( \leftrightarrow ,\Leftrightarrow \), is equivalent to (if and only if, iff)
\end{enumerate}

\begin{multicols}{2}
\paragraph{Negation}
\begin{tabular}{@{ }c | c@{ }@{ }c}
p & \( \neg \) & p\\
\hline
T & \textcolor{red}{F} & T\\
F & \textcolor{red}{T} & F\\
\end{tabular}
\paragraph{Logical And}
\begin{tabular}{@{ }c@{ }@{ }c | c@{ }@{ }c@{ }@{ }c@{ }@{ }c@{ }@{ }c}
p & q &  & p & \( \land\) & q & \\
\hline
T & T &  & T & \textcolor{red}{T} & T & \\
T & F &  & T & \textcolor{red}{F} & F & \\
F & T &  & F & \textcolor{red}{F} & T & \\
F & F &  & F & \textcolor{red}{F} & F & \\
\end{tabular}
\paragraph{Logical Or}
\begin{tabular}{@{ }c@{ }@{ }c | c@{ }@{ }c@{ }@{ }c@{ }@{ }c@{ }@{ }c}
p & q &  & p & \( \lor \) & q & \\
\hline
T & T &  & T & \textcolor{red}{T} & T & \\
T & F &  & T & \textcolor{red}{T} & F & \\
F & T &  & F & \textcolor{red}{T} & T & \\
F & F &  & F & \textcolor{red}{F} & F & \\
\end{tabular}
\paragraph{Implication}
\begin{tabular}{@{ }c@{ }@{ }c | c@{ }@{ }c@{ }@{ }c@{ }@{ }c@{ }@{ }c}
p & q &  & p & \(\rightarrow\) & q & \\
\hline
T & T &  & T & \textcolor{red}{T} & T & \\
T & F &  & T & \textcolor{red}{F} & F & \\
F & T &  & F & \textcolor{red}{T} & T & \\
F & F &  & F & \textcolor{red}{T} & F & \\
\end{tabular}
\paragraph{Equivalence}
\begin{tabular}{@{ }c@{ }@{ }c | c@{ }@{ }c@{ }@{ }c@{ }@{ }c@{ }@{ }c}
p & q &  & p & \(\leftrightarrow\) & q & \\
\hline
T & T &  & T & \textcolor{red}{T} & T & \\
T & F &  & T & \textcolor{red}{F} & F & \\
F & T &  & F & \textcolor{red}{F} & T & \\
F & F &  & F & \textcolor{red}{T} & F & \\
\end{tabular}
\end{multicols}
\subsection{Definitions}
When given the proposition \( p\rightarrow q \), \( q\rightarrow p \) is its converse, \( \neg q\rightarrow \neg p \) is its contrapositive and \( \neg p\rightarrow \neg q \) is its inverse. The contrapositive is equivalent to the original proposition and the converse and inverse are also equivalent.

\paragraph{Tautology}
A proposition that is always true (\(p\lor\neg p\)).

\paragraph{Contradiction}
A proposition that is always false (\(p\land\neg p\)).

\paragraph{Contingency}
A proposition that is neither a tautology nor a contradiction.

\paragraph{Logical Equivalence}
\(p\) and \(q\) are logically equivalent if \(p\leftrightarrow q\) is a tautology. The notation for equivalence is typically \(\equiv\).

\end{document}
