% !TeX spellcheck = en_GB
\documentclass[a4paper, 10pt]{article}

\usepackage{fontspec}
\usepackage{amsmath}
\usepackage{multicol}

\defaultfontfeatures{Mapping=tex-text,Scale=MatchLowercase}
\setmainfont{Ubuntu Light}

\title{Calculus}
\author{Max Kasperowski}

\begin{document}
\maketitle
\tableofcontents

\newpage
\section{Derivatives}
\subsection{Definition}

If the derivative is defined to be \( y = f(x) \), then its derivative is \( f'(x) = \lim_{h\to\infty}\frac{f(x+h)-f(x)}{h} \).
\newline

Equivalent notations for the derivative of \( y = f(x) \) are:
\[ f'(x) = y' = \frac{df}{dx} = \frac{dy}{dx} = \frac{d}{dx}(f(x)) = Df(x) \]
\newline

Equivalent notations for the the derivative of \( y = f(x) \) evaluated at \( x = a \) are:
\[ f'(a) = y'\Bigr|_{x=a} = \frac{df}{dx}\Bigr|_{x=a} = \frac{dy}{dx}\Bigr|_{x=a} = Df(a) \]

\subsection{Properties}
The following properties hold where \( f(x) \) and \( g(x) \) are differential functions and \( c \) and \( n \) are any real numbers.

\begin{multicols}{2}
    \begin{enumerate}
        \item \( (cf)' = cf'(x) \)
        \item \( (f \pm g)' = f'(x) \pm g'(x) \)
        \item \( (fg)' = f'g + fg' \)
        \item \( \left(\frac{f}{g}\right)' = \frac{f'g - fg'}{g^2} \)
        \item \( \frac{d}{dx}(c) = 0 \)
        \item \( \frac{d}{dx}\left(x^n\right) = nx^{n-1} \)
        \item \( \frac{d}{dx}\left(f\left(g(x)\right)\right) = f'\left(g(x)\right)g'(x) \)
    \end{enumerate}
\end{multicols}
\subsection{Common Derivatives}
\begin{multicols}{2}
    \begin{itemize}
        \item[] \( \frac{d}{dx}(x) = 1 \)
        \item[] \( \frac{d}{dx}(\sin{x}) = \cos{x} \)
        \item[] \( \frac{d}{dx}(\cos{x}) = -\sin{x} \)
        \item[] \( \frac{d}{dx}(\tan{x}) = \sec^2x \)
        \item[] \( \frac{d}{dx}(\sec x) = \sec x\tan x \)
        \item[] \( \frac{d}{dx}(\csc x) = -\csc x\cot x \)
        \item[] \( \frac{d}{dx}(\cot x) = -\csc^2 x \)
        \item[] \( \frac{d}{dx}\left(\sin^{-1}x\right) = \frac{1}{\sqrt{1-x^2}} \)
        \item[] \( \frac{d}{dx}\left(\cos^{-1}x\right) = -\frac{1}{\sqrt{1-x^2}} \)
        \item[] \( \frac{d}{dx}\left(\tan^{-1}x\right) = \frac{1}{1+x^2} \)
        \item[] \( \frac{d}{dx}\left(a^x\right) = a^x\ln(a) \)
        \item[]  \( \frac{d}{dx}\left(e^x\right) = e^x \)
        \item[] \( \left(\ln(x)\right) = \frac{1}{x}, x > 0 \)
        \item[] \( \frac{d}{dx}(ln\lvert x\rvert) = \frac{1}{x}, x\neq0 \)
        \item[] \( \frac{d}{dx}\left(\log_a(x)\right) = \frac{1}{x\ln a}, x > 0 \)
    \end{itemize}
\end{multicols}
\subsection{Higher Order Derivatives}
\paragraph{Second Order Derivative}
\[ f''(x) = f^{(2)}(x) = \frac{d^2f}{dx^2} = \left(f'(x)\right)' \]
\paragraph{n\(^{\mathrm{th}}\) Order Derivative}
\[ f^{(n)} = \frac{d^nf}{dx^n} = \left(f^{(n-1)}(x)\right)' \]
\subsection{Implicit Differentiation}
Implicit differentiation is done just as regular differentiation just that care must be taken to keep in mind that \(y\) is a function and therefore the chainrule must be applied when deriving.
\end{document}
